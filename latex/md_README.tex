\chapter{Trabajo-\/2-\/\+LRU}
\hypertarget{md_README}{}\label{md_README}\index{Trabajo-\/2-\/LRU@{Trabajo-\/2-\/LRU}}
\label{md_README_autotoc_md0}%
\Hypertarget{md_README_autotoc_md0}%
Trabajo 2 Estructura de datos Para compilar el programa\+: gcc main.\+c -\/o lru -\/Wall o tambien se puede usar makefile (make)

Para ejecutar el programa\+: ./lru con sus respectivos argumentos

./lru create $<$max\+\_\+capacity$>$ Crear memoria cache con tamaño máximo. ./lru add $<$char$>$ Agregar un dato a la cache. ./lru all Mostrar todos los datos en cache. ./lru get $<$char$>$ Marcar dato como usado (mover al frente). ./lru exit Salir del programa.

En caso de posible error eliminar la carpeta lru\+\_\+cache que se crea en el mismo directorio del proyecto, esto se puede hacer con el comando\+: rm -\/rf lru\+\_\+cache

Toda la documentacion fue hecha en formato Doxygen 