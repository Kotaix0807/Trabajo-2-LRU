
\section{Introduccion}
El presente informe describe el desarrollo e implementación de un sistema de manejo de caché denominado Least Recently Used (LRU), cuyo objetivo es optimizar el uso de memoria reemplazando los datos menos utilizados recientemente.

Este proyecto fue desarrollado en lenguaje C como parte del curso Estructuras de Datos, utilizando estructuras como () para gestionar el almacenamiento y la prioridad de los datos.

A través de esta implementación se busca comprender el funcionamiento de un algoritmo de reemplazo de caché y aplicar conceptos fundamentales como el manejo dinámico de memoria, el uso de punteros y la eficiencia algorítmica.
\newpage
\section{Objetivos}
\begin{itemize}
    \item Comprender el funcionamiento del algoritmo LRU (Least Recently Used).
    \item Implementar estructuras de datos abstractas como [] para la gestión del caché.
    \item Desarrollar habilidades en programación en lenguaje C, especialmente en el manejo de memoria y punteros.
    \item Simular el comportamiento de un sistema de caché limitado en tamaño, que reordene y elimine datos de acuerdo con su uso reciente.
\end{itemize}
\newpage
\section{Comandos}
\subsection{Lru create N}
Este es un ejemplo de contenido de un informe parcial para el proyecto de \textbf{Estructura de datos}
Inicialmente es importante saber lo básico: para usar negrita usamos el comando \textbf{textbf} y para cursiva usamos \textit{textit}, para incluir código en línea usamos \texttt{texttt}.

\subsection{Numeración} % Esto sería un subtítulo
A continuación se hace un ejemplo de numeración:

\begin{enumerate}
    \item En una enumeración uno puede dejar claro que los elementos se suceden unos detrás de los otros.
    \item Este sería el segundo elemento
    \item Este sería el tercer elemento
    \item Este sería el cuarto elemento qut tiene dos elementos adicionales
    \begin{enumerate}
        \item Este sería el primer elemento de la enumeración anidada
        \item Este sería el segundo elemento de la enumeración anidada
    \end{enumerate}
    \item Este sería el quinto elemento
\end{enumerate}

\subsection{Itemización}
Lo anterior también se puede hacer con puntos en lugar de numeración:

\begin{itemize}
    \item En una itemización uno puede listar elementos sin necesidad de un orden.
    \item Este sería el segundo elemento
    \item Este sería el tercer elemento
    \item Este sería el cuarto elemento qut tiene dos elementos adicionales que si son numerados
    \begin{enumerate}
        \item Este sería el primer elemento de la enumeración anidada
        \item Este sería el segundo elemento de la enumeración anidada
    \end{enumerate}
    \item Este sería el quinto elemento
\end{itemize}

\subsection{Código fuente}
Otra cosa que podrían necesitar es la inclusión de código fuente como el mostrado en el Listing \ref{lst:codigo} (note que en el PDF la numeración es automática gracias a los\texttt{labels}):

\begin{lstlisting}[style=CodeStyle, language=C, caption={Este es un codigo fuente}, label={lst:codigo}]
#include <stdio.h>

int main() {
    printf("Hola mundo!");
    return 0;
}
\end{lstlisting}

\subsection{Imagen}
Otra cosa que puede desear hacer es incluir imágenes lo que se hace como se muestra en la figura \ref{fig:imagen}.
\begin{figure}[!ht]
    \centering % Para que aparezca centrada
    \includegraphics[width=0.5\textwidth]{./src/images/umag.png} % Esta linea incluye la imagen
    \caption{Esta es una imagen} % Este es el mensaje que acompaña la imagen
    \label{fig:imagen} % Esta es la referencia a la imagen
\end{figure}

\subsection{Referencias bibliográficas}
Lo primero necesario para incluir una referencia bibliográfica es agregar una entrada en el archivo \texttt{referencias.bib} con el formato \texttt{key = value}, a continuación se muestra un ejemplo de entrada:

\lstinputlisting[style=codestyle, caption={Entrada bibliográfica}, label={lst:referencias}]{./referencias.bib}  % Esta es una forma de incluir un archivo

Las partes de la entrada que se muestran en el listing \ref{lst:referencias} son las siguientes:
\begin{itemize}
    \item \textbf{title}: Título de la referencia
    \item \textbf{author}: Autor(es) de la referencia (Si son muchos separarlos con la palabra \texttt{and})
    \item \textbf{url}: URL de la referencia
    \item \textbf{year}: Año de publicación
    \item \textbf{month}: Mes de publicación
    \item \textbf{day}: Día de publicación
    \item \textbf{urldate}: Fecha en que visitó la referencia
\end{itemize}
Estos campos corresponden a una referencia de tipo \texttt{@online} que es para citar una fuente en línea, si va a citar un libro u otra fuente que no sea online se usaría \texttt{@book} o \texttt{@article} con los campos que corresponda.

Una vez que tenemos la entrada creada existen diferentes maneras de referenciarla:
\begin{itemize}
    \item Si se desea que aparezca el nombre del autor/a y el año de publocación se usa \texttt{cite{key}} lo que se ve así: \cite{Ejemplo}.
    \item Si se hace una mención al nombre se usa \texttt{textcite{key}} lo que se ve así: \textcite{Ejemplo}.
    \item Si se hace una referencia al texto pero no se menciona al autor se puede usar \texttt{parencite{key}} lo que se ve así: \parencite{Ejemplo}.
    \item Se puede hacer una cita al pie de página con \texttt{footcite{key}} lo que se ve así\footcite{Ejemplo}.
\end{itemize}

\subsection{Notas al pie de página}
Es posible también hacer notas al pie de página con el comando \texttt{footnote}\footnote{Esta es una nota al pie de página.}. Note que toda la numeración se hace automáticamente.